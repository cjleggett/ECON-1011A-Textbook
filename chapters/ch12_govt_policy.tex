\chapter{Government Policy and Externalities}
We have primarily examined what might what be called ``free market economies,'' where agents act without any government policy. In particular, we have acted under the assumption that one agent's consumption and behavior cannot affect another agent's behavior except through the market mechanisms. However, in the real world, certain behaviors impose externalities upon other agents. In order to examine the policy responses to these externalities, we need to understand how to model government policymaking, taking into account how agents will respond to the policy, and what types of objectives governments might have. 


\section{Modeling Government Policy} \label{sec:govt_policy}
Our first goal is to understand how to model and optimize government policies. The basic structure for optimizing government policies is as follows:
\begin{enumerate}
    \item Decide which policy variables the government will be able to manipulate (taxes, subsidies, etc).
    \item Agents in the economy optimize their behavior taking the government's policy variables as exogenous. 
    \item The government chooses the policy variables to optimize some objective function, taking the agent's responses as given. 
\end{enumerate}

\subsection*{Maximizing tax revenue}
We will make these steps concrete with a simple example of maximizing tax revenue via a tax on firms. Often in our political discourse, it is assumed that you are able to raise revenue simply by raising taxes. While this is often the case, because taxes distort behavior, the higher the taxes, the less firms will produce and hence it will decrease tax revenues. To see that this would be the case, suppose that governments can only tax revenue (not profits), and the government levies a 100\% tax on revenues. Because firms have other costs, they will choose to simply not produce, which means that the government's tax revenue would be 0. This is commonly known as the \vocab{Laffer curve} argument, which was popularized by economist Arthur Laffer, and was used during the Reagan administration (as well as subsequent administrations) to argue that cutting taxes would in fact raise revenue. 

Here, we will consider the problem of how to determine the optimal tax, by following the above steps.

\begin{enumerate}
    \item First, we will determine the government's policy variable. We will let $\tau$ be a tax on revenues, so that if the exogenous price of the good is $p$, and the firm produces $Q$ units of the good, the government receives,
    \begin{align*}
        \tau p Q
    \end{align*}
    Notice that we could also interpret this as a percentage tax on each unit of the good sold. This type of tax is known as an \vocab{\emph{ad valorum}}, or proportional, tax. 
    \item Next, we solve the firm's profit maximization problem. The firm takes $\tau$ as given, has cost function $C(Q)$, and hence solves,
    \begin{align*}
        \max_{Q} (1 - \tau) p Q - C(Q)
    \end{align*}
    The familiar first order condition defines the optimal production,
    \begin{align*}
        (1 - \tau) p = C'(Q^*)
    \end{align*}
    \item Finally, we maximize the government's objective function, which in this case is tax revenue. We take the firm's choice of production, $Q^*$ So the government solves,
    \begin{align*}
        \max_{\tau} \tau p Q^*(\tau)
    \end{align*}
    This yields first order conditions for the optimal tax, $\tau^*$, 
    \begin{align*}
        \tau^* \frac{dQ^*}{d\tau} = -Q^*
    \end{align*}
    Note that this does not depend directly on price, as multiplying by price is just a monotonic transformation of the objective function. However, price may enter in implicitly through $Q^*$ and $\frac{dQ^*}{d\tau}$. Dividing through by $-Q^*$ yields
    \begin{align*}
        -\frac{\tau^*}{Q^*} \frac{dQ^*}{d\tau} = 1
    \end{align*}
    Notice that on the left, we have the elasticity of $Q$ with respect to $\tau$, which is the elasticity of production with respect to the tax. This tells us that at an optimum, the elasticity of supply with respect to the tax rate must equal 1. Intuitively, if we increase taxes by $1\%$ and our quantity (hence revenue) decreases by $1\%$, then we cannot earn any additional revenue by changing the tax because the decrease in revenue exactly balances out the increase in per unit tax. 
\end{enumerate}

\section{Modeling Externalities} \label{sec:modeling_externalities}
In all of the models that we have considered so far, we have assumed that each agent's utility depends only on the decisions that they make. We showed in the last chapter that in a setting where agents have full information and each agent's choices only directly affects their own utility, then a competitive equilibrium is efficient. In this section, we will examine how to model \vocab{externalities}, which is when the choices of one agent enter directly into the utility function of another agent. Perhaps the most common example of an externality is pollution. When you drive a car, the carbon dioxide emitted from the car not only affects your utility, but the utility of every individual. While externalities are normally thought of as negative, there are also positive externalities, like volunteering to clean up a local park making the park going experience for everyone else better. Of particular interest will be how the government can intervene in the case of externalities to make a Pareto improvement, that is, everyone will be better off than without government intervention. 

It is important to note however, that when we refer to externalities, we mean only when the choice of one agent enters \emph{directly} into the utility of another agent. We do not account for how an agent's choices can affect other agents through the market mechanism. For example, if you decide to buy oranges, increasing the demand for oranges and hence raising the price, and hence decreasing the utility for other orange buyers. While your choices in this case may have affected the utility of other agents, they did so only via market effects rather than direct effects, and hence we do not treat them the same as externalities with direct effects. 

\subsection*{Simple pollution model}
One of the canonical examples of an externality involves pollution. In the case of pollution, individuals gain utility from consuming some pollution generating good. For example, people can gain utility from driving a car, turning on the air conditioner, or charging your laptop, but all of this consumption involves producing carbon. The carbon then has a negative effect on the utility of other individuals in society through air pollution and climate change. 

The key is that the choices of each agent enters into the utility of every other agent. To see how this might be modeled, consider the following simple model of pollution:
\begin{itemize}
    \item There are $N$ identical agents.
    \item Agents have income $Y$.
    \item Agents choose between consuming a numeraire good $X$ and a carbon producing good $C$, where the utility from $C$ is strictly increasing and concave. The price of $X$ is 1, and the price of $C$ is exogenously given as $p$. 
    \item The carbon producing good generates negative externalities, where each agent's utility is negatively related to the average amount of carbon generated by the other agents. 
    \item The utility for agent $j$ is given by:
    \begin{equation*}
        X_j + U(C_j) - k \frac{\sum_{i \neq j} C_i}{N - 1}
    \end{equation*}
    where $k > 0$ is some constant denoting the marginal harm from pollution. 
    \item The government levies an \emph{ad valorum} (per unit) tax $\tau$, and the tax revenue is rebated to citizens in a lump sum, $L$, where $L = \frac{\sum_{i = 1}^N \tau C_i}{N}$. However, we assume that citizens do not internalize the effect of their taxes on $L$. \footnote{This means that agents treat $L$ as exogenous. They do not maximize their utility knowing that consuming more carbon will result in slightly higher tax rebates. This is realistic in the case where $N$ is very large, as any individual's contributions to the total tax revenue will have a negligible effect on their tax rebate.}
    \item Assume that the government's objective is to maximize average utility. Since all $N$ individuals are identical, this also will maximize the utility of every citizen. 
\end{itemize}
Our goal will be to choose the optimal tax rate, $\tau^*$ that maximizes the average citizen's utility. Recall from section \ref{sec:govt_policy} that to solve a government optimization problem, we first maximize the utility of each individual treating the government's choice variables as exogenous. 

\subsubsection*{The individual's maximization problem}
From the perspective of individual $j$, taxes, the lump sum rebate, and the consumption choices of other individuals $i$ is exogenous. Individual $j$'s maximization problem is therefore:
\begin{equation*}
    \max_{X_j, C_j} X_j + U(C_j) - k \frac{\sum_{i \neq j} C_i}{N - 1} \text{ s.t. } X_j + (p + \tau) C_j = Y + L
\end{equation*}
Substituting for the budget constraint\footnote{Substitute in $X_j = Y + L - p C_j$.} yields the unconstrained maximization problem: 
\begin{equation*}
    \max_{C_j} Y + L - (p + \tau) C_j + U(C_j) - k \frac{\sum_{i \neq j} C_i}{N - 1}
\end{equation*}
The only remaining choice variable for the the individual is now $C_j$. Taking first order conditions with respect to $C_j$ yields
\begin{align*}
    p + \tau = U'(C_j)
\end{align*}
Let $C^*$ denote the quantity that satisfies the above first order conditions. Note that in this case, $L = \tau C^*$ since every individual contributes the same amount in taxes and just receives an equal amount in tax rebates. Then each individual's utility in equilibrium is given by
\begin{equation*}
    \begin{split}
        Y + L - (p + \tau) C^* + U(C^*) - k \frac{\sum_{i \neq j} C^*}{N - 1} &= Y + \tau C^* - (p + \tau) C^* + U(C^*) - k \frac{(N - 1)C^*}{N - 1} \\
        &= Y + U(C^*) - (p + k) C^*
    \end{split}
\end{equation*}

\subsubsection*{Government optimization with individual choices as functions}
From the perspective of the government, $C^*$ is a function of the chosen tax rate, $\tau$, which we will denote as $C^* = C(\tau)$ for simplicity. The government chooses taxes to maximize the utility of the average citizen in equilibrium. So, the government's maximization problem is,
\begin{equation*}
    \max_{\tau} Y + U(C(\tau)) - (p + k) C(\tau)
\end{equation*}
Differentiating with respect to $\tau$ to obtain first order conditions yields,
\begin{equation*}
    \left[U'(C(\tau)) - (p + k)\right]C'(\tau) = 0
\end{equation*}
It can be shown that $C'(\tau) \neq 0$ for any choice of $\tau$. So, the first order condition is only satisfied when
\begin{equation*}
    U'(C(\tau)) = p + k
\end{equation*}
While it might seem difficult at first to obtain a closed form solution for $\tau$ that satisfies the above equation, recall that $C(\tau)$ is \emph{defined} such that,
\begin{equation*}
    U'(C(\tau)) = p + \tau
\end{equation*}
Then to satisfy the first order condition, we must have $p + k = p + \tau$. This implies that the optimal tax rate is given by $\tau^* = k$. 

Notice that in the above model, government is able to make a strict Pareto improvement when $k > 0$. That is, so long as pollution has a negative effect on the utilities of other agents, the government can intervene and make everyone better off. This is important because it demonstrates a key failing of the First Welfare Theorem, namely that a competitive equilibrium may not be Pareto efficient if externalities are present. 

\subsection*{Maximizing Firm Profits with Externalities}
To obtain some further intuition about externalities and government policy under externalities, we will examine a canonical model of pollution involving two firms. Firm $A$ will be a polluter, while firm $B$'s production is affected by $A$'s pollution. For simplicity, we will assume that the price of both firms' goods is $p$. 
\begin{itemize}
    \item Firm $A$ makes candy, producing pollution as a byproduct. The firm chooses the quantity of candy, which we will denote by $A$, to produce. Firm $A$'s cost function is given by
    \begin{equation*}
        \frac{1}{2} c_a A^2
    \end{equation*}
    Assume that firm $A$ does not care about the impact of their pollution. The firm maximizes profit, which is given by
    \begin{equation*}
        \pi_A = pA - \frac{1}{2} c_a A^2
    \end{equation*}
    Solving the first order conditions yields the optimal choice of production as, 
    \begin{equation*}
        A^* = \frac{p}{c_a}
    \end{equation*}

    \item Firm $B$ cleans laundry, and laundry is made more costly by pollution. $B$'s cost function is given by
    \begin{equation*}
        \frac{1}{2} c_b B^2 + x A \text{ for $x > 0$ }
    \end{equation*}
    Firm $B$'s profit is therefore:
    \begin{equation*}
        \pi_B = pB - \frac{1}{2}c_b B^2 - x A
    \end{equation*}
    Since $B$ cannot affect $A$'s pollution, pollution can be treated essentially as a fixed cost for $B$. The first order condition yields $B$'s optimal production as
    \begin{equation*}
        B^* = \frac{p}{c_b} 
    \end{equation*}
\end{itemize}
Because we are dealing with two firms in this case, we can make sense of adding their profits to gain a measure of the total ``social'' profit. The total combined profit when firms optimize individually is therefore given by:
\begin{equation*}
    \begin{split}
        \pi_A^* + \pi_B^* &= \frac{p^2}{c_a} - \frac{p^2}{2 c_a} + \frac{p^2}{c_b}  - \frac{p^2}{2 c_b} - x \frac{p}{c_a} \\ 
        &= \frac{p (p - 2x) }{2 c_a} + \frac{p^2}{2 c_b}
    \end{split}
\end{equation*}

\subsubsection*{Joint Profit Optimization}
We can next consider what would happen if firms $A$ and $B$ were to merge into one firm. The goal for the new owner would then be to choose production quantities $A$ and $B$ to maximize total profit.
\begin{equation*}
    \pi_{A + B} = pA - \frac{1}{2} c_a A^2 + pB - \frac{1}{2} c_b B^2 - xA
\end{equation*}
Solving for the optimal quantities yields,
\begin{equation*}
    A^{**} = \frac{p - x}{c_a} \text{ and } B^{**} = \frac{p}{c_b}
\end{equation*}
Notice that $A^{**} < A^*$. This is because the pollution externality is now being internalized by the joint firm. That is, the cost of pollution now affects the decision maker's profit rather than affecting another firm, which effectively increases the cost of producing $A$, hence reducing the quantity produced. Notice however that $B^{**} = B^*$, because in both cases, the cost of pollution is effectively a fixed cost from the perspective of choosing how much $B$ to produce, and hence the optimal quantities are the same. 

Under joint optimization, the total profit becomes
\begin{equation*}
    \pi_{A + B}^{**} = \frac{(p - x)^2}{2 c_a} + \frac{p^2}{2 c_b}
\end{equation*}
First, even before doing any actual calculations, we know that the total profits under joint optimization must be at least the profits under separate optimization. This is by definition because we have chosen $A^{**}$ and $B^{**}$ to optimize the total profit. To determine the exact difference, we can simply subtract the two profit terms from each other to obtain,
\begin{equation*}
    \pi_{A + B}^{**} - (\pi_A^* + \pi_B^*) = \frac{x^2}{2c_a}
\end{equation*}
This is positive for any $x \neq 0$, which is precisely the case where there exists an externality from one firm to another. 

If we treat the total firm profits as some measure of social welfare, this model suggests that total social welfare can be improved when the firms are combined than when they are separate in the case of externalities. If externalities between the two firms did not exist, then there should be no difference in total profit between jointly optimizing and separately optimizing firms. This suggests that if we could somehow enforce that the firms behave as if they were jointly optimizing, then we could improve total social welfare. We will discuss how government policymakers can try to achieve this optimal social welfare without forcing firms to jointly optimize in the next section. 

\section{Pigouvian Taxes and Coase's Theorem}
In the previous section, we showed how to model externalities, where the decisions of one agent can directly affect the utility of another agent. We showed that not only can there be societal losses in efficiency when externalities are present, but the equilibrium outcome may fail to even be pareto efficient. In this section, we offer a more general theoretical examination of government responses to externalities via taxation as well as an examination of the conditions under which a free-market economy can be efficient even with externalities present. In particular, we will describe how governments can use taxes to correct for externalities, and under what conditions a free-market economy can result in a pareto efficient outcome even when externalities are present. 

\subsection*{Pigouvian Taxes}
Recall the two firm model of externalities described in section \ref{sec:modeling_externalities}. We showed that when one firm's pollution affects the profit of another, total profits can be strictly higher when the firms combine than when they optimize separately. The intuition is that when the firms are combined, the costs of pollution are internalized by the polluting firm (as there is only one firm). This suggests that when externalities are present, the government can improve total social welfare by forcing individuals to internalize the cost of their choices on social welfare. A straightforward way to impose such costs is by imposing taxes.

A \vocab{Pigouvian tax} sets taxes so that an individual's costs of taking an action are exactly equal to the social welfare costs of that action\footnote{Pigouvian taxes are named after 20th century economist Arthur Pigou, who first described the concept of taxing externalities in his book, \emph{The Economics of Welfare}, in 1920.}. By setting such a tax, individuals optimize their decisions while taking into account the social cost of their actions. In the case of our firm model, a Pigouvian tax would set a tax $\tau = x$ on each unit of $A$ produced. The cost to firm $A$ of production would therefore include the social cost of the externality. In such a case, the marginal cost of producing $A$ to the firm is exactly the same as if the two firms were merged. Accordingly, the socially efficient amount of $A$ will be produced.  

There are many areas where Pigouvian taxes are evident in the real world. A common proposal for addressing climate change is to impose a carbon tax to discourage the use of carbon producing technologies. However, many Pigouvian taxes do not take the form of an explicit tax. A toll road, for example, can be considered a tax on traffic and pollution by imposing a cost for driving, which contributes to both. There are often also fines on littering, which also imposes a cost on actions that create externalities for other individuals. 

\subsubsection*{What happens to tax revenue?}
When we say that a tax can improve total profits, we are referring not to post-tax profits, but to pre-tax profits. Effectively, we assume that the government is able to rebate the taxes to firms. However, we need to be careful in how tax revenue is rebated and the assumptions we make regarding how individuals perceive the tax rebate. Suppose, for example, that the government just rebated the tax revenue to firm $A$. However, this is essentially the same as the government not levying a tax at all. Firm $A$ would know that any amount that it paid in taxes would be rebated, which would mean that the taxes have no effect on the firm's behavior, and hence would not provide any incentive not to pollute. 

The taxes could also be rebated in an equal lump sum to all individuals, but this also would not fully address the problem in a two firm model. Since there are only two firms, the effective tax on pollution for firm $A$ would be halved as it receives back half of the revenue in the rebate. 

This leaves two options for economists. The first is to take the approach that we did with the simple pollution model, which is to assume that agents will not internalize their effect on the tax rebate. This is a reasonable assumption in the case that the number of agents is assumed to be large. The second approach is to assume that the rebate will be paid only to the agents affected by pollution. In the case of the firm model, this would amount to a transfer from firm $A$ to firm $B$ from the pollution. This second interpretation suggests that an alternative to government intervention would be for firm $A$ to simply compensate firm $B$ for the pollution. We will explore this notion more in the section on Coase's theorem. 

While the discussion of how taxes are returned to the agents may seem like merely a modeling consideration, this can be important in the construction of Pigouvian taxes in the real world. For example, raising the gas tax is often viewed as a way of imposing a Pigouvian tax on carbon and traffic. However, some oppose the gas tax because it is considered a regressive tax, where lower income individuals who need to drive to work will pay a higher share of their income than wealthier individuals. It may be tempting to give a tax break to lower income individuals in order to help them pay for gas, but such an intervention would negate the incentive effects of discouraging gas usage among lower income individuals. Instead, it may be preferable to rebate the revenue from a gas tax through a lump sum transfer to offset the regressive nature of such taxes.

\subsubsection*{Taxes vs quantity regulation}
Instead of Pigouvian taxes, an equivalent option may be to simply regulate actions that generate externalities directly. Rather than taxing pollution, for example, we could simply regulate the amount of pollution that an individual can create. If the government were able to determine the socially optimal amount of pollution for each individual, then restricting the quantity of regulation would achieve the same result as imposing a Pigouvian tax.

However, there are many reasons why this can be difficult in practice. Imposing a socially efficient quantity regulation requires that the government be able to understand the objective functions of every individual and set a regulation specific to that individual. It may be more realistic for governments to calculate the marginal social cost of pollution and impose a uniform tax that is equivalent to that marginal cost. Pigouvian taxes also offer a continuous incentive to reduce one's externalities. In the case of pollution, for example, it may offer an incentive to invest in non-polluting technologies in order to reduce long term costs. 

There is one method of quantity regulation that can achieve many of the same incentive effects of a tax: \vocab{tradable permits}. If a government wants to cap the total amount of pollution in the economy, they can issue pollution permits that limits the amount of pollution an individual can undertake. If these permits can be traded on a market, then there is effectively a market-determined marginal cost for pollution and allows for pollution to be efficiently allocated among individuals for whom pollution technologies yield the greatest marginal benefit. 

\subsection*{Coase's Theorem}
The key to Pigouvian taxes is that individuals internalize the social cost of their actions via a tax on those actions. However, one may ask whether it is really necessary that the government impose a cost on externalities via taxes. Perhaps it would be possible for other individuals to privately impose costs for externalities.

Consider our two firm example from earlier, where firm $A$'s production process resulted in pollution that negatively affects firm $B$. Suppose that $B$ has the legal right to prevent $A$ from polluting. Perhaps $B$ owns the property where $A$ would be polluting, and therefore has the legal right to prohibit $A$'s pollution. While $B$ could just outright ban $A$'s pollution, that would not actually maximize $B$'s profit. Instead, $B$ could charge $A$ for its pollution.

Suppose that $B$ charges $A$ precisely the cost per unit of pollution. In this case, $B$ effectively imposes a Pigouvian tax on $A$ for polluting. Since the $A$'s optimization does not depend on who imposes the cost, then we know that the resulting total profit will be the same as in the case of a government imposed Pigouvian tax. An efficient outcome has been achieved without any need for government intervention.

However, this outcome relied upon $B$ having the right to prevent $A$'s pollution. Suppose instead that $A$ had the right to pollute. A similar outcome could be achieved if $B$ offers to pay $A$ to reduce their pollution. This imposes an \emph{opportunity cost} on $A$ for pollution, as every unit of pollution sacrifices payments that could have been obtained from $B$. While the \emph{total} profits in this case will be the same as the previous case and the Pigouvian tax case, the \emph{distribution} of profits heavily favors $A$. In both this case and the previous case, however, the overall outcome is Pareto efficient, in the sense that neither firm can be made better off without making the other firm worse off. 

The ability of private firms to achieve Pareto efficient outcomes by imposing private costs on each other through bargaining is the key insight behind \vocab{Coase's Theorem}.
\begin{theorem*}[Coase's Theorem]
    If there are no transaction costs, then bargaining leads to a Pareto efficient outcome regardless of the initial endowments. 
\end{theorem*}

The key limitation to the above is that every agent is able to costlessly negotiate with every other affected agent. While this may be satisfied in the simple example with two firms, it may be much more difficult to negotiate something like large scale environmental pollution, where billions of individuals worldwide are affected. In such a case, it seems much more likely that there would be heavy transaction costs and hence Coase's theorem would not hold.

While few economists would argue that the conditions for Coase's theorem hold in the real world, it nonetheless suggests that externalities alone are not sufficient to justify government involvement. Rather than viewing the government as uniquely able to address the problems of externalities, we can instead interpret the government as a coordination mechanism for imposing a general solution to externalities rather than requiring individuals to negotiate with each other. 

One additional consideration is that Coase's theorem only implies that that the outcome after bargaining is Pareto efficient. It does not imply that the resulting distribution of outcomes is equitable or desirable.